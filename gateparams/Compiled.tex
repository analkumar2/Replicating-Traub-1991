\documentclass{article}
\usepackage{amsmath}
\usepackage{graphicx}
\graphicspath{ {./assignment_images/} }

%\renewcommand{\thesubsection}{\alph{subsection}}

\title{L9 Q2}
\date{October 13, 2018}
\author{Anal Kumar}

\begin{document}

\maketitle
\pagenumbering{gobble}

\section*{Consider how you can get a signalling pathway that responds only when synaptic input is in a certain intermediate range. At low frequency synaptic input it does not respond, and at high frequency it again does not respond. Think in terms of the frequency of synaptic input translating into calcium levels.}

A pathway which can respond to low and intermediate calcium levels which corresponds to low frequency of synaptic input makes use of positive cooperativity which inhibits activity. Vice versa, a pathway which can respond to high and intermediate calcium levels which corresponds to high frequency of synaptic input makes use of positive cooperativity which facilitate activity.

A pathway which responds only to intermediate frequency and calcium levels is an intersection of both these pathways.

\includegraphics[width=\linewidth]{L9_Q2_signalling}

Molecule F give rise to activity in the presence of B and D\textsuperscript{*}. B gets inhibited at high concentration of calcium due to positive cooperativity of activation of C with calcium. D\textsuperscript{*} on the other hand increases in the presence of high calcium. Thus only at intermediate concentration of calcium, there is neuronal activity.

\end{document}